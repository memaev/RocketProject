\documentclass {report}

\usepackage [utf8]{inputenc}
\usepackage [russian]{babel}
\usepackage [papersize={15cm, 25cm}]{geometry}
\usepackage {amsmath, amssymb}

\renewcommand{\le}{\leqslant} % <=
\renewcommand{\ge}{\geqslant} %>=

\begin {document}

\begin {center}
{\Large \bf
Вывод формул
}
\end {center}

\bigskip

1) Новый импульс ракеты после выброса продуктов горения: 
\begin{equation}
M \vec{v} + m(\vec{v} + \vec{u}) = (M_0 - m) \vec{v} + m(\vec{v} + \vec{u}) = M_0\vec{v}+ m\vec{u},
\end{equation}
где $M$ --- текущая масса ракеты, $M_0$ --- начальная масса ракеты, $u$ --- скорость продуктов горения относительно ракеты => относительно земли $(v+u)$

\bigskip

2)Также, исходя из закона сохранения импульса, производная полученного импульса по времени равна сумме действующих на ракету сил (силы тяжести и силы сопротивления воздуха), продифференцировав полученный импульс, можно написать следующее: 
\begin{equation}
M_0\dot{\vec{v}} + \dot{m}\vec{u} = (M_0 - m)\vec{g} - \frac{1}{2}C_xS\rho\, v\,\vec{v},
\end{equation}
где $S$ - площадь Миделя ракеты (наибольшая площадь сечения), $C_x$ - коэффициент сопротивления: для носа ракеты $C_x=0.5$, для ракеты с парашютом $C_x=1$, $\rho$ - плотность воздуха на данной высоте (рассчитывается по формуле $\rho_0*e^{-0.0001\cdot y}$)

\bigskip

3) Вектор скорости $\vec{u}$ продуктов горения направлен против вектора $\vec{v}$ и равен по модулю $u_0$, значит его можно представить вот в таком виде:
\begin{equation}
\vec{u} = -u_0\frac{\vec{v}}{v},
\end{equation}
где $\frac{\vec{v}}{v}$ - это единичный вектор с направлением вектора $\vec{v}$

\bigskip

4) Также можно добавить некоторые зависимости \\
Зависимость производной массы топлива по времени:
\begin{equation}
\dot{m} = 
 \begin{cases}
   \dot{m}_0 \mbox{ при } t \le \frac{m_f}{\dot{m_0}} \\
   0 \mbox{ при } t > \frac{m_f}{\dot{m_0}}
 \end{cases}
\end{equation}
где $t$ - текущее время полета, $m_f$ - изначальная масса топлива, $\dot{m_0}$ - скорость расхода топлива (производная массы топлива по времени).


Зависимость массы ракеты по времени:
\begin{equation}
M(t) = 
\begin{cases}
M_0 = \dot{m_0}t \mbox{ при } t \leqslant \frac{m_f}{\dot{m_0}} \\
M_0 - m_f
\end{cases}
\end{equation}
где $t$ - текущее время полета, $m_f$ - изначальная масса топлива, $\dot{m_0}$ - скорость расхода топлива (производная массы топлива по времени).

\bigskip

5) Из формулы 2 можно выразить $\vec{v}$ и расписать по Ox и Oy:
$$
\dot{v_x} = \frac{\dot{m}}{M_0}\cdot u_0 \cdot \frac{v_x}{\sqrt{x^2 + y^2}} - \frac{C_x\cdot S\rho(x; y)}{M_0} \cdot \sqrt{v_x^2 + v_y^2} \cdot v_x
$$
$$
\dot{v_y} = -\frac{M(t)}{M_0} \cdot g + \frac{\dot{m}}{M_0}\cdot u_0 \cdot \frac{v_x}{\sqrt{x^2 + y^2}} - \frac{C_x\cdot S\rho(x; y)}{M_0} \cdot \sqrt{v_x^2 + v_y^2} \cdot v_y
$$

\bigskip
\begin {center}
{\Large \bf
Написание программы
}
\end {center}

\bigskip

1) Наша ракета имеет 3 основных режима полета:
\begin{enumerate}
\item Активный (топливо есть, ракета набирает скорость)
\item Пассивный (топлива нет, ракета сначала движется вверх по инерции, а потом переходит в свободное падение)
\item Полет с парашютом (топлива нет, ракета равномерно спускается с парашютом; $C_x=1, S=5, V=const$, т.к. мы учитываем сопротивление воздуха $X$, которое пропорционально квадрату скорости)
\end{enumerate}

\bigskip

2) В программе для расчета параметров полета, написанной на языке программирования C++ были реализованы две главные функции: f() и step(). \\\\
Функция $f()$ выполняет расчет производных $\dot{x}, \dot{y}, \dot{v_x}, \dot{v_y}$ исходя из данных $x, y, v_x, v_y$, которые передаются в функцию $f()$ в качестве аргументов и формул, вывод которых был показан на прошлом слайде.

Изначальные данные величин: $x=0, y=0, v_x=1, v_y=2$ хранятся в массиве aCurrent. Тангенс угла атаки (угол траектории ракеты к горизонту) определяется отношением $\frac{v_y}{v_x}$. 
\\\\
Функция step() обновляет величины в соответствии с шагом по времени $h_t=0.0025 s$ и работает на основе метода Рунге-Кутта 4-го порядка для более точных расчетов. Функция пошагово интегрирует производные $\dot{x}, \dot{y}, \dot{v_x}, \dot{v_y}$ и переходит к $(n+1)$-му шагу по следующему принципу :
$$
\vec{k_1} = \vec{f}(t_n, \vec{w_n})
$$
$$
\vec{k_2} = \vec{f}(t_n+\frac{1}{2}h_t, \vec{w_n}+\frac{1}{2}h_t\cdot \vec{k_1})
$$
$$
\vec{k_3} = \vec{f}(t_n + \frac{1}{2}h_t, \vec{w_n}+\frac{1}{2}h_t\cdot \vec{k_2})
$$
$$
\vec{k_4} = \vec{f}(t_n + h_r, \vec{w_n} + h_t\cdot \vec{k_3})
$$
$$
\vec{w_{n+1}} = \vec{w_n} + \frac{1}{6}h_t\cdot (\vec{k_1} + 2\vec{k_2} + 2\vec{k_3} + \vec{k_4})
$$
$$
t_{n+1} = t_n + h_t
$$

\end {document}
